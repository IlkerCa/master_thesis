% einleitung.tex
\chapter{Einleitung}
%https://www.freshplaza.de/article/14835/uebersicht-weltmarkt-tomaten/
%https://www.crowdai.org/challenges/plantvillage-disease-classification-challenge
Die weltweit am meisten verzehrte Gemüsesorte ist die Tomate\cite{FreshPlaza} Tomaten werden schon seit mehreren 100 Jahren angebaut und sind in Europa als Gemüsesorte sehr stark vertreten. Im Jahr 2016 lag die globale Produktion von Tomaten bei 177 Millionen und wuchs im 10-Jahres-Zeitraum um 33\%. In der Landwirtschaft haben Tomaten vor allem für viele europäische Länder, zum Beispiel Italien und Spanien, eine große wirtschaftliche Bedeutung. 

Mithilfe von neuen Technologien können Tomatenpflanzen in Mengen angebaut werden, die für eine Nahrungsmittelversorgung von Milliarden Menschen ausreicht. Allerdings treten immer mehr Pflanzenkrankheiten aufgrund der sich verändernden Umweltbedingungen auf, die dieses Angebot bedrohen. Jedes Jahr geht daher ein Großteil der Pflanzen durch Krankheiten verloren, so dass der Lebensunterhalt von weltweit vielen Bauern davon abhängt, wie gut ihre Ernte sein wird. Der Kontinent Afrika ist besonders von dieser Situation betroffen, da 80\% der landwirtschaftlichen Güter von Kleinbauern abstammen\cite{crowdai}.



\section{Motivation und Hintergrund}

Es existieren Milliarden von Smartphones rund um den Globus, die zu einem Werkzeug für die Krankheitsdiagnose werden könnten\cite{crowdai}. Die korrekte, schnelle Erkennung einer Krankheit ist ausschlaggebend für eine rechtzeitige, effiziente Behandlung, die ohne das Einschicken einer Probe in ein Labor auskommt. Das Potenzial, welches von Faltungsnetzen ausgeht, kann genutzt werden, um Modelle zu erstellen und diese auf dem Smartphone auszuführen. Die Techniken aus dem Bereich der tiefen neuronalen Netze können erstaunliche Ergebnisse aufweisen, da die Menge an Daten und die Performanz der Hardware von Jahr zu Jahr steigen.

Pflanzenkrankheiten unterscheiden sich stark darin, wie sie die Blätter befallen. Diese verschiedenen Ausprägungen können genutzt werden, um die Krankheiten voneinander zu unterscheiden.




\section{Struktur der Arbeit}


Diese Masterarbeit behandelt Pflanzenkrankheiten, die bei Tomaten auftreten. Mithilfe von Convolutional Neural Networks sollen diese erkannt werden. Zunächst wird in dem Kapitel \ref{chapter:grundlagen} die Tomate als Pflanze vorgestellt. Dort werden die Herkunft, die Anbauansprüche und die wirtschaftliche Bedeutung der Tomate thematisiert. Anschließend werden in einem weiteren Unterkapitel vier Krankheiten aufgeführt, die für Tomaten schädlich sind. Da die Erkennung der Krankheiten mit neuronalen Netzen erfolgen soll, werden in dem Kapitel \ref{sec:machine_learning} die Grundlagen von neuronalen Netzen erläutert. Anknüpfend wird eine neue Klasse von neuronalen Netzen, nämlich Faltungsnetze, vorgestellt. Diese sind in der Lage, maschinell Bilddateien zu verarbeiten. In dem Hauptteil (Kap. \ref{chapter:entwurf}) werden die Anforderungen der Modelle und die Herangehensweisen beschrieben. Nach der Beschreibung folgt das Kapitel Implementierung (Kap. \ref{chapter:implementierung}). Dort wird die Implementierung der Modelle dokumentiert. Im Schlussteil werden die Ergebnisse der Modelle zusammengefasst und Visualisierungen der Schichten in den neuronalen Netzen veranschaulicht. Im Fazit wird die Arbeit zusammengefasst und Schlussfolgerungen gezogen.


\section{Ziele}

Im Rahmen dieser Masterarbeit sollen Modelle entworfen werden, die mithilfe einer Klassifizierung Krankheitsbefälle bei Tomatenpflanzen erkennen sollen. Die vorliegenden Bilddaten können genutzt werden, um Modelle aus dem Bereich der Faltungsnetze zu erstellen. Verschiedene Ansätze bei der Erstellung von Modellen werden hierbei verfolgt. Des Weiteren wird ein Voting-Klassifizierer für die Erkennung von zwei Pflanzenkrankheiten, die sich in der Ausprägung der Symptome ähneln, erstellt. Außerdem sollen die Aktivierungswerte der Schichten visualisiert werden, um ein besseres Verständnis bezüglich der neuronalen Netze zu erlangen.

