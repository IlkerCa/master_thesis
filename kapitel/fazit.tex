% kapitel2.tex
\chapter{Zusammenfassung und Ausblick}
\label{chapter:fazit}

In diesem Kapitel wird die Masterarbeit zusammengefasst und ein Ausblick auf mögliche
Verbesserungen gegeben.

\section{Zusammenfassung}
In dieser Masterarbeit wurde mehrere Modelle erfolgreich erstellt, die spezifische Pflanzenkrankheiten anhand von Bilddateien erkennen können. Die grundlegende Methodik für die Erstellung der Modelle sind Faltungsnetze. Hierbei wurden mehrere Ansätze verfolgt, um diese zu erstellen. Für besondere Anwendungsszenarien, zum Beispiel die Erkennung von ähnlichen Krankheiten, wurden mehrere Modelle realisiert, die miteinander zusammenarbeiten, um Fehlklassifizierungen zu reduzieren. Des Weiteren wurden die Schichten des Hauptmodells visualisiert, um ein Verständnis der Aktivierungen in dem neuronalen Netz zu erhalten. Für jede Krankheit sowie gesunde Blätter wurden Visualisierungen erstellt. Die Implementierung wurde anschließend dokumentiert. Vor der Implementierung wurden Bibliotheken miteinander verglichen und bewertet, um die Anforderungen der Implementierung zu erfüllen.

\section{Ausblick}
Die Masterarbeit hat gezeigt, dass Faltungsnetze für die Klassifizierung von Pflanzenkrankheiten geeignet sind. Dennoch gibt es Verbesserungsbedarf hinsichtlich der Modelle. Die leichte Überanpassung könnte mit mehr Variationen von Bilddateien angegangen werden. Des Weiteren ist die Leistung des transfergelernten Modells nicht sehr hoch, so dass diese mit mehr Bilddateien verbessert werden kann. Außerdem gibt es eine weitere Möglichkeit, um vortrainierte Modelle nutzen zu können. Diese Möglichkeit nennt sich auch Feintuning. Hierbei wird nur ein Teil der Gewichte von dem vortrainierten Modell statt alle Gewichte benutzt. Weiterhin besteht das Potenzial, die Gewichte der Modelle für andere Nutzpflanzen mit derselben Krankheit zur Verfügung zu stellen. Mithilfe von Transfer-Lernen können die Ergebnisse anderer Datensätze verbessert werden, da die Klassendomänen ähnlich sind. 


